% !TEX TS-program = pdflatex
% !TEX encoding = UTF-8 Unicode

% This is a simple template for a LaTeX document using the "article" class.
% See "book", "report", "letter" for other types of document.

\documentclass[11pt]{article} % use larger type; default would be 10pt

\usepackage[utf8]{inputenc} % set input encoding (not needed with XeLaTeX)

%%% Examples of Article customizations
% These packages are optional, depending whether you want the features they provide.
% See the LaTeX Companion or other references for full information.

%%% PAGE DIMENSIONS
\usepackage[margin=1in]{geometry} % to change the page dimensions
\geometry{a4paper} % or letterpaper (US) or a5paper or....
% \geometry{margin=2in} % for example, change the margins to 2 inches all round
% \geometry{landscape} % set up the page for landscape
%   read geometry.pdf for detailed page layout information

\usepackage{graphicx} % support the \includegraphics command and options

% \usepackage[parfill]{parskip} % Activate to begin paragraphs with an empty line rather than an indent

%%% PACKAGES
\usepackage{booktabs} % for much better looking tables
\usepackage{array} % for better arrays (eg matrices) in maths
\usepackage{paralist} % very flexible & customisable lists (eg. enumerate/itemize, etc.)
\usepackage{verbatim} % adds environment for commenting out blocks of text & for better verbatim
\usepackage{subfig} % make it possible to include more than one captioned figure/table in a single float
% These packages are all incorporated in the memoir class to one degree or another...

%%% HEADERS & FOOTERS
\usepackage{fancyhdr} % This should be set AFTER setting up the page geometry
\pagestyle{fancy} % options: empty , plain , fancy
\renewcommand{\headrulewidth}{0pt} % customise the layout...
\lhead{}\chead{}\rhead{}
\lfoot{}\cfoot{\thepage}\rfoot{}

%%% SECTION TITLE APPEARANCE
\usepackage{sectsty}
\allsectionsfont{\sffamily\mdseries\upshape} % (See the fntguide.pdf for font help)
% (This matches ConTeXt defaults)

%%% ToC (table of contents) APPEARANCE
\usepackage[nottoc,notlof,notlot]{tocbibind} % Put the bibliography in the ToC
\usepackage[titles,subfigure]{tocloft} % Alter the style of the Table of Contents
\renewcommand{\cftsecfont}{\rmfamily\mdseries\upshape}
\renewcommand{\cftsecpagefont}{\rmfamily\mdseries\upshape} % No bold!

\usepackage[backend=biber,style=chem-acs]{biblatex}
\addbibresource{refs.bib}
%\bibliographystyle{plain} % We choose the "plain" reference style
%\bibliography{refs} % Entries are in the refs.bib file

\usepackage{amsmath}
\usepackage{braket}

\usepackage{hyperref}
\hypersetup{colorlinks=true, citecolor=blue, urlcolor=blue, linkcolor=blue}
\usepackage{cleveref}	
  	\crefname{figure}{Figure}{Figures}
  	\crefname{table}{Table}{Tables}
  	\crefname{equation}{Eq.}{Eqs.}
  	\crefname{section}{Section}{Sections}
  	\crefname{subsection}{Section}{Sections}
  	\crefname{subsubsection}{Section}{Sections}
  	\crefname{algorithm}{Algorithm}{Algorithms}

%%% END Article customizations

%%% The "real" document content comes below...

\title{A Few Things I Learned About Tensor Product Methods in Quantum Chemistry}
\author{Samuel Powell}
\date{2021-12-29} % Activate to display a given date or no date (if empty),
         % otherwise the current date is printed 

\begin{document}
\maketitle

\section{Tensor product states}

\subsection{Intro}
A quantum mechanical wavefunction can be represented as a tensor, $T$. Some quantum chemical methods, such as 
DMRG (Density Matrix Renormalization Group) theory approximates this tensor as a product of $N$ order-2 and -3 tensors.
%linked by an hyperdimension $D$.
Tensor product (TP) methods are very successful in describing structurally simple, 
strongly correlated systems.

\subsection{Occuptation number representation}
A tensor product state is most easily described using occupation number representation, in which
a determinant is represented as a string of numbers indicating the occupancy of each orbital. 
In the spin-orbital formalism, where occupation numbers can be either 0 or 1, a ground-state 
configuration may be written

\begin{equation}
    \ket{1111000000\dots}
\end{equation}
where the first four orbitals are occupied and the rest are unoccupied. 

\subsection{Tensor product states}
A general Fock-space state written using the occupancy-number notation appears as:

\begin{equation} \label{mps_state}
    \ket{\psi_{\text{TP}}} = \sum_{n_1,n_2,\dots,n_M}c_{n_1,n_2,\dots,n_M} \ket{n_1n_2\dots n_M} 
\end{equation}
Note that this is a general Fock-space state, as there is no restriction on the number of particles in any term. 
In TPS methods, the total wavefunction is represented as a tensor composed 
of these coefficients, setting equal to zero all coefficients
corresponding to states with a number of electrons not consistent with the physical system being considered. 
In a system of $M$ orbitals, there are $2^M$ such coefficients. 
%bond dimension: a measure of the level of complexity representable by the tensor network. 
In matrix product state (MPS) methods, this $M$-mode tensor is approximated as a 
network of $M$ 2- and 3-order tensors:

\begin{equation}
    \bar{C}_{n_1,n_2,...,n_M} \approx \sum_{b_1,b_2...b_{M-1}}  w^{1}_{n_1,b_1} w^{2}_{b_1,n_2,b_2} \dots w^{M-1}_{b_{M-2},n_{M-1},b_{M-1}} w^{M}_{b_{M-1},n_M}
\end{equation}
where each $w^i$ is an order-2 or order-3 tensor, and there are $M-1$ contractions over the set 
$\set{b^i}$ of internal indices. The decomposition is a purely mathematical approximation,
and the resultant low-order tensors do not have an intuitive physical meaning, although the
goal of the decomposition is naturally to minimize the lost entanglement or electronic structure information. 

This notation allows for a compact representation of any arbitrary state in Fock space. 
% in contrast to the determinant formulation, as in \cref{ci_wf}, that more efficiently describe HF or 
%perturbative methods, which describe a wavefunction relative to some reference state.
% The energies of the TP wavefunction % would be $o^M0$?
The Density Matrix Renormalization Group (DMRG) algorithm, which variationally optimizes MPS tensors,
is one of the most prominent tensor network methods in electronic structure theory.
\supercite{white_density_1992, white_density-matrix_1993, white_ab_1999}

TP methods are very efficient in determining low-energy states of 1-D systems, but the
rapidly increasing complexity of the tensor network representation for 3-D or otherwise complicated
electronic states generally makes study of most 3-D systems intractable. \supercite{chan_density_2011} 
In spite of its efficiency, the
computational complexity severely limits the possible applications. 
Extending TPS methods to higher-dimensional systems is an active area of research,
and many of these methods can describe strongly correlated systems. 
\supercite{jordan_classical_2008, kovyrshin_self-adaptive_2017, marti_complete-graph_2010, murg_tree_2015, szalay_tensor_2015, nakatani_efficient_2013, abraham_selected_2020, abraham_cluster_2021}

\section{Further reading and influences}
If you'd like to know more about tensor product state methods, you can just get bent, because it no one seems to 
have written a comprehensible and comprehensive description of them; I read books and papers for a while,
but ended up synsthesizing this from a mash-up of what I could understand and remember from the comments of 
3 different people with Ph.D.s, all of whom, of course, gave very different descriptions. 

\printbibliography

\end{document}
