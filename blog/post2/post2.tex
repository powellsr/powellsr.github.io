% !TEX TS-program = pdflatex
% !TEX encoding = UTF-8 Unicode

% This is a simple template for a LaTeX document using the "article" class.
% See "book", "report", "letter" for other types of document.

\documentclass[11pt]{article} % use larger type; default would be 10pt

\usepackage[utf8]{inputenc} % set input encoding (not needed with XeLaTeX)

%%% Examples of Article customizations
% These packages are optional, depending whether you want the features they provide.
% See the LaTeX Companion or other references for full information.

%%% PAGE DIMENSIONS
\usepackage[margin=1in]{geometry} % to change the page dimensions
\geometry{a4paper} % or letterpaper (US) or a5paper or....
% \geometry{margin=2in} % for example, change the margins to 2 inches all round
% \geometry{landscape} % set up the page for landscape
%   read geometry.pdf for detailed page layout information

\usepackage{graphicx} % support the \includegraphics command and options

% \usepackage[parfill]{parskip} % Activate to begin paragraphs with an empty line rather than an indent

%%% PACKAGES
\usepackage{booktabs} % for much better looking tables
\usepackage{array} % for better arrays (eg matrices) in maths
\usepackage{paralist} % very flexible & customisable lists (eg. enumerate/itemize, etc.)
\usepackage{verbatim} % adds environment for commenting out blocks of text & for better verbatim
\usepackage{subfig} % make it possible to include more than one captioned figure/table in a single float
% These packages are all incorporated in the memoir class to one degree or another...

%%% HEADERS & FOOTERS
\usepackage{fancyhdr} % This should be set AFTER setting up the page geometry
\pagestyle{fancy} % options: empty , plain , fancy
\renewcommand{\headrulewidth}{0pt} % customise the layout...
\lhead{}\chead{}\rhead{}
\lfoot{}\cfoot{\thepage}\rfoot{}

%%% SECTION TITLE APPEARANCE
\usepackage{sectsty}
\allsectionsfont{\sffamily\mdseries\upshape} % (See the fntguide.pdf for font help)
% (This matches ConTeXt defaults)

%%% ToC (table of contents) APPEARANCE
\usepackage[nottoc,notlof,notlot]{tocbibind} % Put the bibliography in the ToC
\usepackage[titles,subfigure]{tocloft} % Alter the style of the Table of Contents
\renewcommand{\cftsecfont}{\rmfamily\mdseries\upshape}
\renewcommand{\cftsecpagefont}{\rmfamily\mdseries\upshape} % No bold!

\usepackage[backend=biber,style=chem-acs]{biblatex}
\addbibresource{refs.bib}
%\bibliographystyle{plain} % We choose the "plain" reference style
%\bibliography{refs} % Entries are in the refs.bib file

\usepackage{amsmath}
\usepackage{braket}

\usepackage{hyperref}
\hypersetup{colorlinks=true, citecolor=blue, urlcolor=blue, linkcolor=blue}
\usepackage{cleveref}	
  	\crefname{figure}{Figure}{Figures}
  	\crefname{table}{Table}{Tables}
  	\crefname{equation}{Eq.}{Eqs.}
  	\crefname{section}{Section}{Sections}
  	\crefname{subsection}{Section}{Sections}
  	\crefname{subsubsection}{Section}{Sections}
  	\crefname{algorithm}{Algorithm}{Algorithms}

%%% END Article customizations

%%% The "real" document content comes below...

\title{Blithering on Observed Social Phenomena I Don't Understand Or Even Know Exist}
\author{Samuel Powell}
\date{2022-01-06} % Activate to display a given date or no date (if empty),
         % otherwise the current date is printed 

\begin{document}
\maketitle

\section{The issue}
There seems to be a pattern of smart, self-sufficient people in technical fields (to be specific on what I have observed,
 programming), choosing to be explicit and crude in their use of language in personal communications (blogs).
 
 \section{Examples}
 Ran across this blog page trying to debug a problem at work. It's about limits of \verb|std::size_t| in C++. See \url{https://www.duskborn.com/posts/on-the-use-and-abuse-of-size_t/} \href{https://www.duskborn.com/posts/on-the-use-and-abuse-of-size_t/}{this  blog}.
 
 I use the term blog a lot here, probably mostly because of the previous example, which I came across the week I wrote this. 
 I should probably find more examples, but I've come across enough for this to stick in my head. Another example is
 \href{https://www/motherfuckingwebsite.com}{this site}. This one is satire, which could explain some of it. The condescending 
 tone is certainly an important part of the satire, but you could probably avoid expletives. 
 
 \section{Futher thoughts}
 I think what perturbs me is the crass tone and vocabulary in what seems like a semi-professional setting. 
 
 \section{Speculative explanations}
 I think there's a movement toward a more casual workplace and more personal expressions in the public sphere and in 
 professional settings. Not sure why. 
 
 
 
 \section{Caveats}
 I am rather socially conservative and generally have a conservative and polished public presentation through language. 
 I express myself in socially appropriate ways, and avoid explicit references and crude language (generally). This means 
 I just may not know what is normal.
 
 Additionally, I may be falsely correlating technical knowledge (specifically in programming or technology fields) with this 
 phenomena. People in these fields are more likely to \emph{have} blogs in which this may be observed. I'm also much more likely 
 to see these blogs in particular, given my interests and career position.
 
 


\printbibliography

\end{document}
